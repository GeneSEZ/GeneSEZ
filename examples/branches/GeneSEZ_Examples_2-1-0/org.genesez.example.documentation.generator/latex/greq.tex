
\subsection{Class {\tt RModel}}
Represents the entire requirements model, i.e. the root element of the model tree.

\subsubsection{Attributes}
The class \verb|RModel| defines the following attributes:
\begin{description}
	\item[name : String] The name of the requirements model, usually the same name as the project the requirements contained by the model are used for.
\end{description}

\subsubsection{References}
The class \verb|RModel| defines the following references:
\begin{description}
	\item[requirement : RRequirement] Contains a list of all requirements within the model.
	\item[scenario : RScenario] Contains a list of all scenarios within the model.
	\item[extension : RAnnotation] Contains a list of all annotations defined for this model.
\end{description}

\subsection{Class {\tt RRequirement}}
A requirement is a very focused statement about a paricular business need concerning a paricular unit of a system.

\subsubsection{Attributes}
The class \verb|RRequirement| defines the following attributes:
\begin{description}
	\item[rationale : String] Describes the sense of the requirement and can be useful to find a test context for the requirement.
Usually the \verb|rationale| describes the requirement from another point of view as the \verb|description|.
	\item[type : String] The type is used to classifiy requirements.
Commonly used are \emph{functional} , 'non-functional' and \emph{constraint} .
	\item[precedence : String] Specifies an ordinal value to determine crucial requirements by relating the values of different requirements.
Possible values include the priority or a target relese.
	\item[verificationMethod : String] Specifies how this requirement can be verified.
It should just consits of a single word for an easy possible evaluation.
Possibilities are: \emph{test} , \emph{demo}  and \emph{review} . You can check the SysML specification for more (non normative) possibilities.
\end{description}

\subsubsection{References}
The class \verb|RRequirement| defines the following references:
\begin{description}
	\item[model : RModel] Reference to the requirements model.
	\item[parent : RRequirement] Reference to the parent requirement. The opposite is child.
It is used to nest requirements, with the exception of namespace embedding.

This dependency is taken from SysML with change that no requirement is embedded into the namespace of one another.
	\item[child : RRequirement] Contains a list of child requirements. The opposite is parent.
It is used to nest requirements, with the exception of namespace embedding.

This dependency is taken from SysML with change that no requirement is embedded into the namespace of one another.
	\item[deriving : RRequirement] Reference to the requirement this requirement is derived from. The opposite is derived.
It can be used to express a dependency between requirements in the means that the existence of one requirements it logical by one another.

This dependency is taken from SysML.
	\item[derived : RRequirement] Contains a list of derived requirements. The opposite is deriving.
It can be used to express a dependency between requirements in the means that the existence of one requirement is logical by one another.

This dependency is taken from SysML.
	\item[refining : RRequirement] Reference to the requirement this requirement is refined from. The opposite is refined.
It can be used to express a dependency between requirements in the means that one requirements is more precise than one another.

This dependency is taken from SysML.
	\item[refined : RRequirement] Contains a list of refined requirements. The opposite is refining.
It can be used to express a dependency between requirements in the means that one requirements is more precise than one another.

This dependency is taken from SysML.
	\item[supportedScenario : RScenario] References to scenarios involving this requirement in their business objective.
The opposite is supportedRequirement.
	\item[supportedScenarioStep : RScenarioStep] References to scenario steps involving this requirement in their business objective.
The opposite is supportedRequirement.
\end{description}

\subsection{Class {\tt RScenario}}
A scenario is a description of how a system can be used to achieve a particular business need.
It consists of steps which need to be performed in the defined order.

A scenario is something like a \emph{use case} or a \emph{user story}.

\subsubsection{Attributes}
The class \verb|RScenario| defines the following attributes:
\begin{description}
	\item[precedence : String] Specifies an ordinal value to determine crucial scenarios by relating the values of different scenarios.
Possible values include the priority or a target relese.
	\item[verificationMethod : String] Specifies how this scenario can be verified.
It should just consits of a single word for an easy possible evaluation.
Possibilities are: \emph{test} , \emph{demo}  and \emph{review} . You can check the SysML specification for more non normative possibilities.
\end{description}

\subsubsection{References}
The class \verb|RScenario| defines the following references:
\begin{description}
	\item[model : RModel] Reference to the requirements model.
	\item[step : RScenarioStep] Contains a list of steps which must be performed in the defined order.
	\item[supportedRequirement : RRequirement] Reference to requirements, involved in performing this scenario. The opposite is supportedScenario.
\end{description}

\subsection{Class {\tt RScenarioStep}}
A scenario step is a particular, elementary doing of an actor involving a system.


\subsubsection{References}
The class \verb|RScenarioStep| defines the following references:
\begin{description}
	\item[scenario : RScenario] Reference to the context this step is performed.
	\item[supportedRequirement : RRequirement] Reference to a list of requirements, involved in performing this step.
\end{description}

\subsection{Class {\tt RAnnotation}}
Defines a particular model extension, similar to an UML stereotype.

\subsubsection{Attributes}
The class \verb|RAnnotation| defines the following attributes:
\begin{description}
	\item[uri : String] Provides a unique identifier for this model extension.
\end{description}

\subsubsection{References}
The class \verb|RAnnotation| defines the following references:
\begin{description}
	\item[tag : RTag] Contains a list of \emph{tags}.
	\item[model : RModel] Reference to the requirement model element.
\end{description}

\subsection{Class {\tt RTag}}
A \verb|tag| represents a named definition of some meta data, like a key in a map or a tag of an UML stereotype.

\subsubsection{Attributes}
The class \verb|RTag| defines the following attributes:
\begin{description}
	\item[name : String] A description for the meta data the \verb|tag| identifies.
	\item[type : String] Optionally identifies the type of the meta data the \verb|tag| identifies.
\end{description}

\subsubsection{References}
The class \verb|RTag| defines the following references:
\begin{description}
	\item[annotation : RAnnotation] References the model extension defining the \verb|tag|.
\end{description}

\subsection{Class {\tt RValue}}
Represents the meta data stored within a model extension, similar to values in a map or UML tagged values.

\subsubsection{Attributes}
The class \verb|RValue| defines the following attributes:
\begin{description}
	\item[value : String] Stores the meta data information.
\end{description}

\subsubsection{References}
The class \verb|RValue| defines the following references:
\begin{description}
	\item[tag : RTag] References the meta data definition.
	\item[object : RSpecObject] References the \emph{specification object} this meta data is assigned to.
\end{description}

\subsection{Class {\tt RSpecObject}}
Represents a specification object.
A specification object is general construct to group common properties of specification descriptions.

\subsubsection{Attributes}
The class \verb|RSpecObject| defines the following attributes:
\begin{description}
	\item[definition : String] A short but meaningful objective of the specification object.
	\item[id : String] The unique identifier of the specification object with repect to this model but not globally unique.
It means that objects of every specialized class can be uniquely identified, but not objects accross more specializations (globally unique).
	\item[version : String] The version of the specification object is usually the date of the last modification or a requirement model global change indicating number like the version number in the subversion version control system.
	\item[url : String] The URL to the corresponding specification item in the requirements management tool.
\end{description}

\subsubsection{References}
The class \verb|RSpecObject| defines the following references:
\begin{description}
	\item[annotation : RAnnotation] Contains a list of all annotations assigned to this specification object.
	\item[value : RValue] Contains a list of all values this specification object is annotated with.
\end{description}

	